\documentclass{article}
\begin{document}
The core proof system is contained entirely within \verb|common/proof.v|
This reduces a reachability claim in any transition relation to
showing that the set of claims can be supported by a circular proof.
The rest and the bulk of the system is proof automation for applying
this system.

The overall skeleton of the tactic is defined in
\verb|implementation/fun_semantics.v|, parameterized over
tactics for domain reasoning, and instantiated in each of the
example directories.

In the broad outline, the generic solver tactic is to break
the claims down into separate proof goals for each distinct
starting configuration, take the initial step according to
the transition system, and then call the generic run tactic
to finish the proof.

The run tactic tries in order to finish immediately
(case \verb|ddone|), to apply some assumed claim through
transitivity (case \verb|dtrans|), to take a step
accordingb to the transition relation (case \verb|dstep|),
or to make a case distinction about a symbolic value on which
execution is blocked, and at any point if execution is
``stuck'' to pause at that point and leave proof goals for
the user to make manual progress on, or the proof automation
designer to automate.

In their own way, each of these cases have to deal to a greater
or lesser degree with deciding what exactly to do,
handling failure, and manipulating hypotheses.

The decision to be made in the \verb|ddone| case is whether
the proof can or should try to conclude from the current state,
with the dual goals of not trying expensive tactics when the
current state couldn't possibly satisfy the goal predicate,
and recognizing when the state looks enough like the goal to
be worth pausing for the user even if automated tactics don't
manage to prove that the goal is satisfied.

This is done using a tactic argument to econstructor -
which is applied to resulting goals in order, with the
results of existential variable instantiations in earlier
goals visible in later goals.

To support this, the type of constructors of the relations
should be structured so that unifying the conclusion with the
goal handles everything that can be handled by Coq matching,
and the remaining hypotheses have side conditions in increasing
order of difficulties --  isValue checks that can be solved
by computation first, then map equivalances that take a
call to equate_maps, and more complicated domain reasoning
last.